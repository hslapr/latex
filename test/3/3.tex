\documentclass{jsarticle}
\usepackage{booktabs}
\usepackage{array}
\begin{document}

\begin{center}
	\begin{tabular}{lrr} \toprule
		品名   & 単価(円) & 個数 \\ \midrule
		りんご & 100        & 5    \\
		みかん & 50         & 10 \\ \bottomrule
	\end{tabular}
\end{center}

\begin{center}
	\begin{tabular}{|l|r|r|} \hline
		品名   & 単価(円) & 個数 \\ \hline
		りんご & 100        & 5    \\ \cline{1-1}
		みかん & 50         & 10 \\ \hline
	\end{tabular}
\end{center}

魔法陣(表\ref{mahou})
\begin{table}
	\caption[3次の魔法陣]{3次の魔法陣の例。
	縦・横・斜めの和がいずれも15である。}
	\label{mahou}
	\begin{center}
		\setlength{\tabcolsep}{3pt}
		\footnotesize
		\begin{tabular}{|c|c|c|} \hline
			2 & 9 & 4 \\ \hline
			7 & 5 & 3 \\ \hline
			6 & 1 & 8 \\ \hline
		\end{tabular}
	\end{center}
\end{table}%
では縦・横・斜めの和が等しい。

\end{document}